\chapter{Sparse Data and Formats}
\section{Definition}

Data are said sparse when it is contains only a few non-null values. That kind of dataset are really common in Machine Learning application and can be an high influence on the computation. 

The sparsity of a dataset is defined by :

\begin{equation}\label{eqn:sparsity}
sparsity = \frac{\text{\# non-null values}}{\text{\# values}}
\end{equation}

Conversely when a dataset has only a few null values, the data are said dense. The density of the dataset is defined by the inverse of the sparsity:
\begin{equation}\label{eqn:density}
density = \frac{1}{\text{sparsity}}
\end{equation}

Using dense methods and data structure with sparse data could have a severe bad impact on the performance
\section{The Advantages of Sparse Data}
Sparsity is a very useful property in Machine Learning. Some algorithms can have fast optimization, fast evaluation of the model, statistical robustess or other computational advantages.
A lot of machine learning application are using sparse dataset such as recommender system, natural language processing algorithm, 

\section{Sparse Data are very common in Machine Learning}

In Machine Learning it's very common to deal with sparse dataset. We can encounter them in any kind of applications: Natural Language Processing, Retrieving Systems, Recommender Systems, etc.

Given the possible optimization that sparse dataset allows and the high number of people that could take advantages of it, it becomes important to add the support of sparse data in Nd4j. 
\subsection{A Real Case of Sparse Dataset}
In 2008 Netflix launched a contest, the Netflix Grand Prize \cite{netflixgrandprize}, to improve their recommender system model and to increase the accuracy of predictions and published an sample dataset made with the ratings of anonymous Netflix customers. The dataset had more than 100 millions sampled ratings and it contained about $m=480'186$ users and $m=17'770$ movies \cite{Koren091the}. If stored as a dense matrix, it would need to store $8'532'905'220$ values in memory. That corresponds to a sparsity $\cong \frac{100'000'000}{8'532'905'220} = 0.011719338$.

Storing more than 8 trillions 64-bit floating-point numbers needs more than 64 gigabyte of memory which quickly become unmanageable even for the world's fastest supercomputers. 

\section{Solution: Encode the data into a Sparse Format}

To avoid the high volume of storage needed issue, we must store the data into a sparse format. There are different kind of formats which each of them is more suitable to different aspects (storage, matrix-matrix operations, vector-matrix operations,...)



\section{Formats}

There exists several different formats to store a sparse array. The idea behind using a sparse format instead of the classic dense one, is to reduce the memory space and the executing time of the operations. Knowing that a matrix is sparse allows to shortcut some operation steps. For example during a matrix multiplication, we can avoid to perform the multiplication for the zero elements of the sparse matrix.

\subsection{Matrices}
\subsubsection{Coordinates Format}

This format is the simplest format to encode a sparse array. The coordinates and the value of each non-zero entry are stored in arrays.
Typically each element are encoded in a tuple (row, column, value)

Some implementation variations of the COO format exist. The elements can be sorted along a dimension, or it can be some duplicate indexes.

\begin{figure}[h]
	\[
	A_{(M\times N)} = 
	\begin{bmatrix}
	0 &  2 & 0 \\
	0 &  0 & 3 \\
	1 &  0 & 4\\
	0 &  0 & 0
	\end{bmatrix}
	\quad\rightarrow\quad
	\begin{aligned}
	Values_{(1\times NNZ)} = 
	\begin{bmatrix}
	2 &  3 & 1 & 4
	\end{bmatrix}
	\\
	Rows_{(1\times NNZ)} = 
	\begin{bmatrix}
	0 &  1 & 2 & 2
	\end{bmatrix}
	\\
	Columns_{(1\times NNZ)} = 
	\begin{bmatrix}
	1 &  2 & 0 & 2
	\end{bmatrix}
	\end{aligned}
	\]
	\caption{A matrix stored in COO format}
\end{figure}

With this format it's easy and fast to retrieve the value given an index and to insert a new non-zero element.. It's also fast and simple to convert into a dense format.

But this format don't minimize the memory space. It can be reduced with a compressed format such as CSR or CSC as described below.

\subsubsection{Compressed Row Format}

The Compressed Row and the Compressed Column formats are the most general format to store a sparse array. They don't store any unnecessary element. But it requires more steps to access the elements than the COO format. 

Each non-zero element of a row are stored contiguously in the memory. Each row are also contiguously stored.

The format requires four arrays:
\begin{description}[leftmargin=!,labelwidth=\widthof{\bfseries Beginning of row pointers}]
	\item [Values] All the nonzero values are store contiguously in an array. The array size is {NNZ}.
	\item [Column pointers] This array keeps the column position for each values.
	\item [Beginning of row pointers] Each pointer $i$ points to the first element of the row $i$ in the values array. The array size is the number of rows of the array.
	\item [End of row pointers]  Each pointer $i$ points to the first element in the values array that does not belong to the row $i$ . The array size is the number of rows of the array.
\end{description}

\begin{figure}
	\[
	A_{(N\times M)} = 
	\begin{bmatrix}
	0 & 2 & 0 & 0\\
	0 & 0 & 3 & 0\\
	0 & 0 & 0 & 0\\
	1 & 0 & 4 & 0\\
	0 & 0 & 2 & 1
	\end{bmatrix}
	\quad\rightarrow\quad
	\begin{aligned}
	Values_{(1\times NNZ)} = 
	\begin{bmatrix}
	2 &  3 & 1 & 4 & 2 & 1
	\end{bmatrix}
	\\
	Columns_{(1\times NNZ)} = 
	\begin{bmatrix}
	1 &  2 & 0 & 2 & 2 & 3
	\end{bmatrix}
	\\
	pointersB_{(1\times N)} = 
	\begin{bmatrix}
	0 & 1 & 2 & 2 & 4 
	\end{bmatrix}
	\\
	PointersE_{(1\times N)} = 
	\begin{bmatrix}
	1 & 2 & 2 & 4 & 6
	\end{bmatrix}
	\\
	\end{aligned}
	\]
	\caption{A matrix stored in CSR format}
\end{figure}
\subsubsection{Compressed Column Format}
The Compressed Column Format is similar to {CSR} but it compresses columns instead of rows.

Given a matrix $N\times M$, the pointers arrays will have a size $M$. 
\subsection{Tensors - Multi-dimensional arrays}

A tensor is a multi-dimensional array. The order of the tensor is the dimensionality of the array needed to represent it. Matrices and vectors can be represented as tensors where the order is equals to 2 and 1 respectively.

This generalization allows a more generic implementation of a n-dimensional array in the Nd4j library.
\subsubsection{Coordinates Format}
The COO format can easily be extended to encode tensors by storing an array of indexes instead the row and column coordinates. 

A array of order $K = 3$ with shape $N\times M \times P$ which has the following non-zero values :
\begin{center}
	\begin{tabular}{ l | c  }
		value & indexes\\ \hline
		1 & 0 1 0\\ \hline
		2 & 1 1 2 \\ \hline
		3 & 1 2 0 \\ \hline
		4 & 2 0 1 \\ \hline
		5 & 2 2 0 \\ 
		
	\end{tabular}
\end{center}

can be encoded with one values array and one indexes array :
\begin{figure}[h]
	\[
	\begin{aligned}
	Values_{(1\times NNZ)} = 
	\begin{bmatrix}
	1, &  2, & 3, & 4, & 5
	\end{bmatrix}
	\\
	Indexes_{(NNZ \times K)} = 
	\begin{bmatrix}
	[0, 1, 0] ,&  [1, 1, 2], & [1, 2, 0], & [2, 0, 1], & [2, 2, 0]
	\end{bmatrix}
	\end{aligned}
	\]
	\caption{A tensor stored in COO format}
\end{figure}
