\chapter{Sparse Formats}

\section{Vectors and Matrices}


There exists several different formats to store a sparse array. The idea behind using a sparse format instead of the classic dense one, is to reduce the memory space and the executing time of the operations. Knowing that a matrix is sparse allows to shortcut some operation steps. For example during a matrix multiplication, we can avoid to perform the multiplication for the zero elements of the sparse matrix.

\subsection{Coordinate format (COO)}

This format is the simplest format to encode a sparse array. The coordinates and the value of each non-zero entry are stored in arrays.
Typically each element are encoded in a tuple (row, column, value)

Some implementation variations of the COO format exist. The elements can be sorted along a dimension, or it can be some duplicate indexes.


\[
A_{(M\times N)} = 
\begin{bmatrix}
0 &  2 & 0 \\
0 &  0 & 3 \\
1 &  0 & 4\\
0 &  0 & 0
\end{bmatrix}
\quad\rightarrow\quad
\begin{aligned}
Values_{(1\times NNZ)} = 
\begin{bmatrix}
2 &  3 & 1 & 4
\end{bmatrix}
\\
Rows_{(1\times NNZ)} = 
\begin{bmatrix}
0 &  1 & 2 & 2
\end{bmatrix}
\\
Columns_{(1\times NNZ)} = 
\begin{bmatrix}
1 &  2 & 0 & 2
\end{bmatrix}
\end{aligned}
\]

With this format it's easy and fast to retrieve the value given an index and to insert a new non-zero element.. It's also fast and simple to convert into a dense format.

But this format don't minimize the memory space. It can be reduced with a compressed format such as CSR or CSC as described below.

\subsection{Compressed Row Format (CSR)}
The Compressed Row and the Compressed Column formats are the most general format to store a sparse array. They don't store any unnecessary element. But it requires more steps to access the elements than the COO format. 

Each non-zero element of a row are stored contiguously in the memory. Each row are also contiguously stored.

The format requires four arrays:
\begin{description}
	\item [values] All the nonzero values are store contiguously in an array. The array size is {NNZ}.
	\item [column pointers] This array keeps the column position for each values.
	\item [Beginning of row pointers] Each pointer $i$ points to the first element of the row $i$ in the values array. The array size is the number of rows of the array.
	\item [End of row pointers]  Each pointer $i$ points to the first element in the values array that does not belong to the row $i$ . The array size is the number of rows of the array.
\end{description}

\[
A_{(N\times M)} = 
\begin{bmatrix}
0 &  2 & 0 & 0\\
0 &  0 & 3 & 0\\
1 &  0 & 4 & 0\\
0 &  0 & 0 & 1
\end{bmatrix}
\quad\rightarrow\quad
\begin{aligned}
Values_{(1\times NNZ)} = 
\begin{bmatrix}
2 &  3 & 1 & 4 & 1
\end{bmatrix}
\\
Columns_{(1\times NNZ)} = 
\begin{bmatrix}
1 &  2 & 0 & 2 & 2
\end{bmatrix}
\\
pointersB_{(1\times N)} = 
\begin{bmatrix}
0 & 1 & 2 & 4 
\end{bmatrix}
\\
PointersE_{(1\times N)} = 
\begin{bmatrix}
1 & 2 & 4 & 5
\end{bmatrix}
\\
\end{aligned}
\]

\section{Tensors / N-dimensional arrays}

\subsubsection{Coordinate format(COO)}
