\chapter{Sparse Formats}

\section{Vectors and Matrices}


There exists several different formats to store a sparse array. The idea behind using a sparse format instead of the classic dense one, is to reduce the memory space and the executing time of the operations. Knowing that a matrix is sparse allows to shortcut some operation steps. For example during a matrix multiplication, we can avoid to perform the multiplication for the zero elements of the sparse matrix.

\subsection{Coordinate format (COO)}

This format is the simplest format to encode a sparse array. The coordinates and the value of each non-zero entry are stored in arrays.
Typically each element are encoded in a tuple (row, column, value)

Some implementation variations of the COO format exist. The elements can be sorted along a dimension, or it can be some duplicate indexes.


\[
A = 
\begin{bmatrix}
0 &  2 & 0 \\
0 &  0 & 3 \\
1 &  0 & 4\\
0 &  0 & 0
\end{bmatrix}
\quad\rightarrow\quad
\begin{aligned}
Values = 
\begin{bmatrix}
2 &  3 & 1 & 4
\end{bmatrix}
\\
Rows = 
\begin{bmatrix}
0 &  1 & 2 & 2
\end{bmatrix}
\\
Columns = 
\begin{bmatrix}
1 &  2 & 0 & 2
\end{bmatrix}
\end{aligned}
\]

With this format it's easy and fast to retrieve the value given an index and to insert a new non-zero element.. It's also fast and simple to convert into a dense format.

But this format don't minimize the memory space. It can be reduced with a compressed format such as CSR or CSC as described below.

\subsection{Compressed Row Format (CSR)}


\section{Tensors / N-dimensional arrays}

\subsubsection{Coordinate format(COO)}
