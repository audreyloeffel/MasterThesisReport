\appendix
\chapter{Sparse Offset - An execution example}

	This appendix presents an execution of the algorithm \ref{alg:sparseOffsets} presented in section \ref{sssec:sparseOff}.
	
	Let's start with an tensor $myTensor$ with a shape equal to $[2, 4, 4, 5]$ on which we are calling the following operation:
	
	\begin{lstlisting}[style=nonumbers]
	myTensor.get(NDArrayIndex.all(),
	NDArrayIndex.point(0),
	NDArrayIndex.point(3),
	NDArrayIndex.all());
	\end{lstlisting}
	
	Assuming that $myTensor$ is a view of a 5-order tensor and has a the following parameters:
	
	\begin{lstlisting}[style=nonumbers]		
	sparseOffsets = [1, 1, 0, 0, 0]
	flags = [1, 0, 0, 0, 0]
	\end{lstlisting}
	
	Assuming we name the dimension as [book, page, row, column], we are taking all the columns of the last row of the first pages of each book.
	
	The indexes resolution returns the following parameters:
	
	\begin{lstlisting}[style=nonumbers]
	offsets = [0, 0]
	shape =  [2, 5]
	offset = 15
	\end{lstlisting}
	
	First step is to iterate over the dimension:\\
	\textbf{Iteration: i=0}\\
	Number of element in one book: $numElement = 4\times 4\times 5 = 60$ \\
	Book offset = $\lfloor offset/numElement\rfloor =  \lfloor 15/60\rfloor = 0$\\  
	Then we update the offset: $offset = offset - 0 * 60$\\
	\textbf{Iteration: i=1}\\
	Number of element in one page = $numElement = 4\times 5 = 20$ \\
	Page offset = $\lfloor offset/numElement\rfloor =  \lfloor 15/20\rfloor = 0$ \\ 
	Then we update the offset: $offset = 15 - 0 * 20$\\
	\textbf{Iteration: i=2}\\
	Number of element in one row = $numElement = 5$\\
	Page offset = $\lfloor offset/numElement\rfloor =  \lfloor 15/5\rfloor = 3$ \\ 
	Then we update the offset: $offset = 15 - 3 * 5 = 0$\\
	
	Finally we reach the last dimension:\\
	Column offset = $offset \mod numElement = 0 \mod 5 = 0$
	
	We get an temporary offsets array equal to $ newOffsets = [0, 0, 3, 0]$. Now we need to merge with the existing offsets of $myTensor$.
	
	Its first dimension is fixed, so we copy its sparse offset :\\ $finalOffest[0] = myTensor.sparseOffset[0] = 1$\\
	Its second dimension is active and there is already an non-zero offset. The offset is equal to\\ $finalOffset[1] = newOffset[0] + myTensor.sparseOffset[1] = 0 + 1 = 1$\\
	Its second dimension is active. The offset is equal to\\ $finalOffset[2] = newOffset[1] + myTensor.sparseOffset[2] = 0 + 0 = 0$\\
	Its third dimension is active. The offset is equal to\\ $finalOffset[3] = newOffset[2] + myTensor.sparseOffset[3] = 3 + 0 = 0$\\
	Its third dimension is active. The offset is equal to\\ $finalOffset[4] = newOffset[3] + myTensor.sparseOffset[4] = 0 + 0 = 0$\\
	
	We finally get the final sparseOffset : $[1, 1, 0, 3, 0]$
